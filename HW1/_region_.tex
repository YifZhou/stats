\message{ !name(hw1.tex)}
\message{ !name(hw1.tex) !offset(-2) }

%%%%%%%%%%%%%%%%%%%%%%%%%%%%%%%%%%%%%%%%% 
% Short Sectioned Assignment
% LaTeX Template
% Version 1.0 (5/5/12)
% 
% This template has been downloaded from:
% http://www.LaTeXTemplates.com
% 
% Original author:
% Frits Wenneker (http://www.howtotex.com)
% 
% License:
% CC BY-NC-SA 3.0 (http://creativecommons.org/licenses/by-nc-sa/3.0/)
% 
%%%%%%%%%%%%%%%%%%%%%%%%%%%%%%%%%%%%%%%%% 

% ----------------------------------------------------------------------------------------
% PACKAGES AND OTHER DOCUMENT CONFIGURATIONS
% ----------------------------------------------------------------------------------------

\documentclass[paper=letter, fontsize=11pt]{scrartcl} % A4 paper and 11pt font size

\usepackage[T1]{fontenc} % Use 8-bit encoding that has 256 glyphs
\usepackage{fourier} % Use the Adobe Utopia font for the document -
%comment this line to return to the LaTeX default
% \usepackage{fontspec}
% \setmainfont{Times}
\usepackage[english]{babel} % English language/hyphenation
\usepackage{amsmath,amsfonts,amsthm} % Math packages
\usepackage{bm}
\usepackage{graphicx}
\usepackage{sectsty} % Allows customizing section commands
%\allsectionsfont{\centering \normalfont\scshape} % Make all sections
%centered, the default font and small caps
\allsectionsfont{\bfseries\rmfamily\large}
\usepackage{fancyhdr} % Custom headers and footers
\pagestyle{fancyplain} % Makes all pages in the document conform to the custom headers and footers
\fancyhead{} % No page header - if you want one, create it in the same way as the footers below
\fancyfoot[L]{} % Empty left footer
\fancyfoot[C]{} % Empty center footer
\fancyfoot[R]{\thepage} % Page numbering for right footer
\renewcommand{\headrulewidth}{0pt} % Remove header underlines
\renewcommand{\footrulewidth}{0pt} % Remove footer underlines
\setlength{\headheight}{13.6pt} % Customize the height of the header

\numberwithin{equation}{section} % Number equations within sections (i.e. 1.1, 1.2, 2.1, 2.2 instead of 1, 2, 3, 4)
\numberwithin{figure}{section} % Number figures within sections (i.e. 1.1, 1.2, 2.1, 2.2 instead of 1, 2, 3, 4)
\numberwithin{table}{section} % Number tables within sections (i.e. 1.1, 1.2, 2.1, 2.2 instead of 1, 2, 3, 4)

\setlength\parindent{0pt} % Removes all indentation from paragraphs - comment this line for an assignment with lots of text

% ----------------------------------------------------------------------------------------
% TITLE SECTION
% ----------------------------------------------------------------------------------------

\newcommand{\horrule}[1]{\rule{\linewidth}{#1}} % Create horizontal rule command with 1 argument of height

\title{ 
  \normalfont \normalsize 
  \textsc{} \\ [25pt] % Your university, school and/or department name(s)
  \horrule{0.5pt} \\ [0.4cm] % Thin top horizontal rule
  \huge STAT HW1 \\ % The assignment title
  \horrule{2pt} \\ [0.5cm] % Thick bottom horizontal rule
}
\author{Yifan Zhou} % Your name
\date{\normalsize\today} % Today's date or a custom date

\begin{document}
\maketitle % Print the title
\section{Random Number Generators}
In class, we talked about many variants of random number generators
and explored their properties in terms of their (i) period, (ii)
clustering, (iii) efficiency, and (iv) portability. In your computer
language and/or algorithm library of choice, choose the random
generator that you will be using for the rest of the class. Search the
literature for articles that explore the above properties of your
generator and summarize your findings in a couple of paragraphs. Make
sure to include references and, if you find it necessary, figures.

\section{Designing surveys}
The Kepler mission has been observing a very large number of stars in
a small patch in the sky and is making a very reliable measurement of
the occurrence rate of planets around solar type stars (see Batalha,
N. M. 2014, Proceedings of the National Academy of Science, 111,
12647; arXiv:1409.1904). For the purposes of this homework problem, we
will assume that the occurrence rate of planets with radii between one
and two Earth radii has been measured to a very high accuracy and it
is equal to 10\% (i.e., 10\% of solar type stars harbor such planets;
the actual rate is consistent with this number but has some
considerable uncertainty).

You are designing a survey of solar type stars in a
different patch of the sky to find the same type of planets. How
many stars would you need to observe in order to have a 90\%
likelihood that you will find at least 30 planets with radii
between one and two Earth radii? What if you want to have a 99\%
likelihood?


\section{Blackbody distribution}
\newcommand{\epsi}{\ensuremath{\epsilon}}
\newcommand{\dd}{\ensuremath{\mathrm{d}}}
The energy distribution of the number of photons that follow the
blackbody distribution is given by

where $\epsi$ is the photon energy, $C$ is a normalization constant, $T$ is the
temperature of the distribution and $k$ is the Boltzmann constant. (Note
that this is the distribution of the number of photons and not of the
radiation energy density, which is the expression that you are
probably more familiar with). Our goal is to generate an ensemble of
photons with energies drawn from this distribution and in a range
$(\epsi_{1}, \epsi_{2})$, using the rejection method. (i) Start by making a change of
variables
\end{document}
\message{ !name(hw1.tex) !offset(-118) }
